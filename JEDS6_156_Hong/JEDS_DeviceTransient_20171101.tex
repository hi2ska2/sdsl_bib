%

%% bare_jrnl.tex
%% V1.4b
%% 2015/08/26
%% by Michael Shell
%% see http://www.michaelshell.org/
%% for current contact information.
%%
%% This is a skeleton file demonstrating the use of IEEEtran.cls
%% (requires IEEEtran.cls version 1.8b or later) with an IEEE
%% journal paper.
%%
%% Support sites:
%% http://www.michaelshell.org/tex/ieeetran/
%% http://www.ctan.org/pkg/ieeetran
%% and
%% http://www.ieee.org/

%%*************************************************************************
%% Legal Notice:
%% This code is offered as-is without any warranty either expressed or
%% implied; without even the implied warranty of MERCHANTABILITY or
%% FITNESS FOR A PARTICULAR PURPOSE! 
%% User assumes all risk.
%% In no event shall the IEEE or any contributor to this code be liable for
%% any damages or losses, including, but not limited to, incidental,
%% consequential, or any other damages, resulting from the use or misuse
%% of any information contained here.
%%
%% All comments are the opinions of their respective authors and are not
%% necessarily endorsed by the IEEE.
%%
%% This work is distributed under the LaTeX Project Public License (LPPL)
%% ( http://www.latex-project.org/ ) version 1.3, and may be freely used,
%% distributed and modified. A copy of the LPPL, version 1.3, is included
%% in the base LaTeX documentation of all distributions of LaTeX released
%% 2003/12/01 or later.
%% Retain all contribution notices and credits.
%% ** Modified files should be clearly indicated as such, including  **
%% ** renaming them and changing author support contact information. **
%%*************************************************************************


% *** Authors should verify (and, if needed, correct) their LaTeX system  ***
% *** with the testflow diagnostic prior to trusting their LaTeX platform ***
% *** with production work. The IEEE's font choices and paper sizes can   ***
% *** trigger bugs that do not appear when using other class files.       ***                          ***
% The testflow support page is at:
% http://www.michaelshell.org/tex/testflow/



\documentclass[journal]{IEEEtran}
%
% If IEEEtran.cls has not been installed into the LaTeX system files,
% manually specify the path to it like:
% \documentclass[journal]{../sty/IEEEtran}





% Some very useful LaTeX packages include:
% (uncomment the ones you want to load)


% *** MISC UTILITY PACKAGES ***
%
%\usepackage{ifpdf}
% Heiko Oberdiek's ifpdf.sty is very useful if you need conditional
% compilation based on whether the output is pdf or dvi.
% usage:
% \ifpdf
%   % pdf code
% \else
%   % dvi code
% \fi
% The latest version of ifpdf.sty can be obtained from:
% http://www.ctan.org/pkg/ifpdf
% Also, note that IEEEtran.cls V1.7 and later provides a builtin
% \ifCLASSINFOpdf conditional that works the same way.
% When switching from latex to pdflatex and vice-versa, the compiler may
% have to be run twice to clear warning/error messages.






% *** CITATION PACKAGES ***
%
\usepackage{cite}
% cite.sty was written by Donald Arseneau
% V1.6 and later of IEEEtran pre-defines the format of the cite.sty package
% \cite{} output to follow that of the IEEE. Loading the cite package will
% result in citation numbers being automatically sorted and properly
% "compressed/ranged". e.g., [1], [9], [2], [7], [5], [6] without using
% cite.sty will become [1], [2], [5]--[7], [9] using cite.sty. cite.sty's
% \cite will automatically add leading space, if needed. Use cite.sty's
% noadjust option (cite.sty V3.8 and later) if you want to turn this off
% such as if a citation ever needs to be enclosed in parenthesis.
% cite.sty is already installed on most LaTeX systems. Be sure and use
% version 5.0 (2009-03-20) and later if using hyperref.sty.
% The latest version can be obtained at:
% http://www.ctan.org/pkg/cite
% The documentation is contained in the cite.sty file itself.






% *** GRAPHICS RELATED PACKAGES ***
%
\ifCLASSINFOpdf
  \usepackage[pdftex]{graphicx}
  % declare the path(s) where your graphic files are
  % \graphicspath{{../pdf/}{../jpeg/}}
  % and their extensions so you won't have to specify these with
  % every instance of \includegraphics
  % \DeclareGraphicsExtensions{.pdf,.jpeg,.png}
\else
  % or other class option (dvipsone, dvipdf, if not using dvips). graphicx
  % will default to the driver specified in the system graphics.cfg if no
  % driver is specified.
  \usepackage[dvips]{graphicx}
  % declare the path(s) where your graphic files are
  % \graphicspath{{../eps/}}
  % and their extensions so you won't have to specify these with
  % every instance of \includegraphics
  % \DeclareGraphicsExtensions{.eps}
\fi
% graphicx was written by David Carlisle and Sebastian Rahtz. It is
% required if you want graphics, photos, etc. graphicx.sty is already
% installed on most LaTeX systems. The latest version and documentation
% can be obtained at: 
% http://www.ctan.org/pkg/graphicx
% Another good source of documentation is "Using Imported Graphics in
% LaTeX2e" by Keith Reckdahl which can be found at:
% http://www.ctan.org/pkg/epslatex
%
% latex, and pdflatex in dvi mode, support graphics in encapsulated
% postscript (.eps) format. pdflatex in pdf mode supports graphics
% in .pdf, .jpeg, .png and .mps (metapost) formats. Users should ensure
% that all non-photo figures use a vector format (.eps, .pdf, .mps) and
% not a bitmapped formats (.jpeg, .png). The IEEE frowns on bitmapped formats
% which can result in "jaggedy"/blurry rendering of lines and letters as
% well as large increases in file sizes.
%
% You can find documentation about the pdfTeX application at:
% http://www.tug.org/applications/pdftex





% *** MATH PACKAGES ***
%
\usepackage{amsmath}
% A popular package from the American Mathematical Society that provides
% many useful and powerful commands for dealing with mathematics.
%
% Note that the amsmath package sets \interdisplaylinepenalty to 10000
% thus preventing page breaks from occurring within multiline equations. Use:
\interdisplaylinepenalty=2500
% after loading amsmath to restore such page breaks as IEEEtran.cls normally
% does. amsmath.sty is already installed on most LaTeX systems. The latest
% version and documentation can be obtained at:
% http://www.ctan.org/pkg/amsmath





% *** SPECIALIZED LIST PACKAGES ***
%
%\usepackage{algorithmic}
% algorithmic.sty was written by Peter Williams and Rogerio Brito.
% This package provides an algorithmic environment fo describing algorithms.
% You can use the algorithmic environment in-text or within a figure
% environment to provide for a floating algorithm. Do NOT use the algorithm
% floating environment provided by algorithm.sty (by the same authors) or
% algorithm2e.sty (by Christophe Fiorio) as the IEEE does not use dedicated
% algorithm float types and packages that provide these will not provide
% correct IEEE style captions. The latest version and documentation of
% algorithmic.sty can be obtained at:
% http://www.ctan.org/pkg/algorithms
% Also of interest may be the (relatively newer and more customizable)
% algorithmicx.sty package by Szasz Janos:
% http://www.ctan.org/pkg/algorithmicx




% *** ALIGNMENT PACKAGES ***
%
%\usepackage{array}
% Frank Mittelbach's and David Carlisle's array.sty patches and improves
% the standard LaTeX2e array and tabular environments to provide better
% appearance and additional user controls. As the default LaTeX2e table
% generation code is lacking to the point of almost being broken with
% respect to the quality of the end results, all users are strongly
% advised to use an enhanced (at the very least that provided by array.sty)
% set of table tools. array.sty is already installed on most systems. The
% latest version and documentation can be obtained at:
% http://www.ctan.org/pkg/array


% IEEEtran contains the IEEEeqnarray family of commands that can be used to
% generate multiline equations as well as matrices, tables, etc., of high
% quality.




% *** SUBFIGURE PACKAGES ***
\ifCLASSOPTIONcompsoc
  \usepackage[caption=false,font=normalsize,labelfont=sf,textfont=sf]{subfig}
\else
  \usepackage[caption=false,font=footnotesize]{subfig}
\fi
% subfig.sty, written by Steven Douglas Cochran, is the modern replacement
% for subfigure.sty, the latter of which is no longer maintained and is
% incompatible with some LaTeX packages including fixltx2e. However,
% subfig.sty requires and automatically loads Axel Sommerfeldt's caption.sty
% which will override IEEEtran.cls' handling of captions and this will result
% in non-IEEE style figure/table captions. To prevent this problem, be sure
% and invoke subfig.sty's "caption=false" package option (available since
% subfig.sty version 1.3, 2005/06/28) as this is will preserve IEEEtran.cls
% handling of captions.
% Note that the Computer Society format requires a larger sans serif font
% than the serif footnote size font used in traditional IEEE formatting
% and thus the need to invoke different subfig.sty package options depending
% on whether compsoc mode has been enabled.
%
% The latest version and documentation of subfig.sty can be obtained at:
% http://www.ctan.org/pkg/subfig




% *** FLOAT PACKAGES ***
%
%\usepackage{fixltx2e}
% fixltx2e, the successor to the earlier fix2col.sty, was written by
% Frank Mittelbach and David Carlisle. This package corrects a few problems
% in the LaTeX2e kernel, the most notable of which is that in current
% LaTeX2e releases, the ordering of single and double column floats is not
% guaranteed to be preserved. Thus, an unpatched LaTeX2e can allow a
% single column figure to be placed prior to an earlier double column
% figure.
% Be aware that LaTeX2e kernels dated 2015 and later have fixltx2e.sty's
% corrections already built into the system in which case a warning will
% be issued if an attempt is made to load fixltx2e.sty as it is no longer
% needed.
% The latest version and documentation can be found at:
% http://www.ctan.org/pkg/fixltx2e


%\usepackage{stfloats}
% stfloats.sty was written by Sigitas Tolusis. This package gives LaTeX2e
% the ability to do double column floats at the bottom of the page as well
% as the top. (e.g., "\begin{figure*}[!b]" is not normally possible in
% LaTeX2e). It also provides a command:
%\fnbelowfloat
% to enable the placement of footnotes below bottom floats (the standard
% LaTeX2e kernel puts them above bottom floats). This is an invasive package
% which rewrites many portions of the LaTeX2e float routines. It may not work
% with other packages that modify the LaTeX2e float routines. The latest
% version and documentation can be obtained at:
% http://www.ctan.org/pkg/stfloats
% Do not use the stfloats baselinefloat ability as the IEEE does not allow
% \baselineskip to stretch. Authors submitting work to the IEEE should note
% that the IEEE rarely uses double column equations and that authors should try
% to avoid such use. Do not be tempted to use the cuted.sty or midfloat.sty
% packages (also by Sigitas Tolusis) as the IEEE does not format its papers in
% such ways.
% Do not attempt to use stfloats with fixltx2e as they are incompatible.
% Instead, use Morten Hogholm'a dblfloatfix which combines the features
% of both fixltx2e and stfloats:
%
% \usepackage{dblfloatfix}
% The latest version can be found at:
% http://www.ctan.org/pkg/dblfloatfix




%\ifCLASSOPTIONcaptionsoff
%  \usepackage[nomarkers]{endfloat}
% \let\MYoriglatexcaption\caption
% \renewcommand{\caption}[2][\relax]{\MYoriglatexcaption[#2]{#2}}
%\fi
% endfloat.sty was written by James Darrell McCauley, Jeff Goldberg and 
% Axel Sommerfeldt. This package may be useful when used in conjunction with 
% IEEEtran.cls'  captionsoff option. Some IEEE journals/societies require that
% submissions have lists of figures/tables at the end of the paper and that
% figures/tables without any captions are placed on a page by themselves at
% the end of the document. If needed, the draftcls IEEEtran class option or
% \CLASSINPUTbaselinestretch interface can be used to increase the line
% spacing as well. Be sure and use the nomarkers option of endfloat to
% prevent endfloat from "marking" where the figures would have been placed
% in the text. The two hack lines of code above are a slight modification of
% that suggested by in the endfloat docs (section 8.4.1) to ensure that
% the full captions always appear in the list of figures/tables - even if
% the user used the short optional argument of \caption[]{}.
% IEEE papers do not typically make use of \caption[]'s optional argument,
% so this should not be an issue. A similar trick can be used to disable
% captions of packages such as subfig.sty that lack options to turn off
% the subcaptions:
% For subfig.sty:
% \let\MYorigsubfloat\subfloat
% \renewcommand{\subfloat}[2][\relax]{\MYorigsubfloat[]{#2}}
% However, the above trick will not work if both optional arguments of
% the \subfloat command are used. Furthermore, there needs to be a
% description of each subfigure *somewhere* and endfloat does not add
% subfigure captions to its list of figures. Thus, the best approach is to
% avoid the use of subfigure captions (many IEEE journals avoid them anyway)
% and instead reference/explain all the subfigures within the main caption.
% The latest version of endfloat.sty and its documentation can obtained at:
% http://www.ctan.org/pkg/endfloat
%
% The IEEEtran \ifCLASSOPTIONcaptionsoff conditional can also be used
% later in the document, say, to conditionally put the References on a 
% page by themselves.




% *** PDF, URL AND HYPERLINK PACKAGES ***
%
%\usepackage{url}
% url.sty was written by Donald Arseneau. It provides better support for
% handling and breaking URLs. url.sty is already installed on most LaTeX
% systems. The latest version and documentation can be obtained at:
% http://www.ctan.org/pkg/url
% Basically, \url{my_url_here}.




% *** Do not adjust lengths that control margins, column widths, etc. ***
% *** Do not use packages that alter fonts (such as pslatex).         ***
% There should be no need to do such things with IEEEtran.cls V1.6 and later.
% (Unless specifically asked to do so by the journal or conference you plan
% to submit to, of course. )


% correct bad hyphenation here
\hyphenation{op-tical net-works semi-conduc-tor}

%\newcommand{\revision}[1]{\textbf{#1}}
\newcommand{\revision}[1]{{#1}}

\begin{document}

%\bstctlcite{IEEEexample:BSTcontrol}

%
% paper title
% Titles are generally capitalized except for words such as a, an, and, as,
% at, but, by, for, in, nor, of, on, or, the, to and up, which are usually
% not capitalized unless they are the first or last word of the title.
% Linebreaks \\ can be used within to get better formatting as desired.
% Do not put math or special symbols in the title.
\title{Transient Simulation of Semiconductor Devices Using a Deterministic Boltzmann Equation Solver}
%
%
% author names and IEEE memberships
% note positions of commas and nonbreaking spaces ( ~ ) LaTeX will not break
% a structure at a ~ so this keeps an author's name from being broken across
% two lines.
% use \thanks{} to gain access to the first footnote area
% a separate \thanks must be used for each paragraph as LaTeX2e's \thanks
% was not built to handle multiple paragraphs
%

\author{Sung-Min~Hong,~\IEEEmembership{Member,~IEEE} and Jae-Hyung Jang,~\IEEEmembership{Member,~IEEE}%
\thanks{This work was supported by the Samsung Research Funding Center of Samsung Electronics under Grant SRFC-IT1401-09.}
\thanks{S.-M. Hong and J.-H. Jang are with the School of Electrical Engineering and Computer Science, Gwangju Institute of Science and Technology (GIST), Gwangju 61005, Korea.
e-mail: smhong@gist.ac.kr}}

% note the % following the last \IEEEmembership and also \thanks - 
% these prevent an unwanted space from occurring between the last author name
% and the end of the author line. i.e., if you had this:
% 
% \author{....lastname \thanks{...} \thanks{...} }
%                     ^------------^------------^----Do not want these spaces!
%
% a space would be appended to the last name and could cause every name on that
% line to be shifted left slightly. This is one of those "LaTeX things". For
% instance, "\textbf{A} \textbf{B}" will typeset as "A B" not "AB". To get
% "AB" then you have to do: "\textbf{A}\textbf{B}"
% \thanks is no different in this regard, so shield the last } of each \thanks
% that ends a line with a % and do not let a space in before the next \thanks.
% Spaces after \IEEEmembership other than the last one are OK (and needed) as
% you are supposed to have spaces between the names. For what it is worth,
% this is a minor point as most people would not even notice if the said evil
% space somehow managed to creep in.



% The paper headers
\markboth{Submitted to IEEE Journal of the Electron Devices Society}{}
%\markboth{Journal of \LaTeX\ Class Files,~Vol.~14, No.~8, August~2015}%
%{Shell \MakeLowercase{\textit{et al.}}: Bare Demo of IEEEtran.cls for IEEE %Journals}
% The only time the second header will appear is for the odd numbered pages
% after the title page when using the twoside option.
% 
% *** Note that you probably will NOT want to include the author's ***
% *** name in the headers of peer review papers.                   ***
% You can use \ifCLASSOPTIONpeerreview for conditional compilation here if
% you desire.




% If you want to put a publisher's ID mark on the page you can do it like
% this:
%\IEEEpubid{0000--0000/00\$00.00~\copyright~2015 IEEE}
% Remember, if you use this you must call \IEEEpubidadjcol in the second
% column for its text to clear the IEEEpubid mark.



% use for special paper notices
%\IEEEspecialpapernotice{(Invited Paper)}




% make the title area
\maketitle

% As a general rule, do not put math, special symbols or citations
% in the abstract or keywords.
\begin{abstract}
   In this paper, the transient simulation of semiconductor devices using a deterministic Boltzmann equation solver is presented.
   Transient simulation capability is implemented in a deterministic Boltzmann equation solver for the three-dimensional momentum space based on the spherical harmonics expansion.    
   The numerical simulation results with implicit time marching methods demonstrate that the transient simulation using a deterministic Boltzmann equation solver can be performed. 
   The impact of the quasi-static approximation for the current density, which is widely adopted in the momentum-based equations, is tested for various devices such as homogeneous samples, an N$^+$NN$^+$ structure, and a MOSFET.
\end{abstract}

% Note that keywords are not normally used for peerreview papers.
\begin{IEEEkeywords}
Semiconductor device modeling, transient simulation, Boltzmann transport equation, deterministic Boltzmann equation solver, spherical harmonics expansion.
\end{IEEEkeywords}






% For peer review papers, you can put extra information on the cover
% page as needed:
% \ifCLASSOPTIONpeerreview
% \begin{center} \bfseries EDICS Category: 3-BBND \end{center}
% \fi
%
% For peerreview papers, this IEEEtran command inserts a page break and
% creates the second title. It will be ignored for other modes.
\IEEEpeerreviewmaketitle

%///////////////
%// Section I //
%///////////////

\section{Introduction}

   \IEEEPARstart{D}{eterministic} Bolzmann equation solvers 
\cite{Gnudi1993,Hennacy1995,Jungemann2006,Jin2008,Hong2009,Hong2011,Ruic2015}, which solve the semiclassical Boltzmann transport equation for carrier transport in a deterministic manner without relying on the Monte Carlo method,
recently have gained popularity in the modeling and simulation society.
   Among various variations of these solvers, for three-dimensional electron gases, the spherical harmonics expansion method is frequently used. 

   Starting from the steady-state simulation using the Gummel iteration method \cite{Gnudi1993}, the full Newton-Raphson method has been introduced \cite{Hong2009, Ruic2015}.
   Introduction of the full Newton-Raphson method enables 
small-signal and noise analyses \cite{Hong2009, Ruic2015}.
   An electrothermal simulation \cite{Kamrani2017} can be easily performed, since all variables are always evaluated.
   It can be used as a valuable tool to simulate the hot-carrier-related problems such as the oxide reliability \cite{Bina2014,Kamrani2017a}.  
   It is expected that deterministic Boltzmann equation solvers will play an increasingly important role in the semiconductor device simulation. 
           
\revision{In spite of the recent progress introduced above, 
a fundamental simulation capability -- transient simulation with an implicit time marching scheme -- is still lacking \cite{Rupp2016}.
   It is in contrast to the Monte Carlo method.
   The Monte Carlo method is inherently transient, therefore, transient responses of the simulated system can be easily obtained. 
   For example, the transient simulation via full-band Monte Carlo simulations for bipolar transistors, inverters, and ring oscillators can be found in \cite{Tiwari1990,Laux1997}.
   However, in the deterministic Boltzmann equation solvers, 
additional effort for the transient simulation is required, because the time derivative term should be separately implemented.}
    
   Although some reports on transient simulation results can be found in the literature \cite{Koseki2016,Di2016}, those works adopt the explicit time marching scheme, which is very different from the implicit time marching scheme used in the conventional device simulators.
   Since the maximum time step for the explicit time marching technique is restricted due to the stability issue, it can be applied only to the very short time duration. 
   Therefore, transient simulation capability using an implicit time marching technique is desirable for practical purposes.

   One application area of particular interest is the simulation of plasma modes in the gated electron gas \cite{Dyakonov1993}.
   Recently, the plasma instability has been simulated by solving the balance equations numerically \cite{Hong2015}.
   In \cite{Kargar2016}, it is reported that the higher-order transport model (for example, the Boltzmann transport equation) is more desirable.
   Therefore, it is very interesting to check whether the plasma effects can be obtained by solving the Boltzmann transport equation directly.

   In this work, transient simulation results using the deterministic Boltzmann equation solver are reported.
   The transient Boltzmann equation solver is applied to the simulation of the plasma effects.
   The organization of the paper is as follows. 
   In Section II, our simulation framework is briefly introduced.   
   In Section III, numerical results for homogeneous samples and an N$^+$NN$^+$ structure are presented.  
   In Section IV, the plasma effects in a MOSFET are tested.
   Our conclusions are presented in Section V. 

%////////////////
%// Section II //
%////////////////

\section{Simulation framework}

\subsection{Stabilization scheme}

   A deterministic Boltzmann equation solver based on the spherical harmonics expansion for the three-dimensional momentum space has been newly implemented into our in-house device simulation framework, G-Device \cite{Hong2015}.
   Three non-parabolic valleys in the conduction band of silicon are considered.
   The phonon scattering mechanisms and their parameters are taken from \cite{Hong2011}.
   The isotropic-elastic approximation for the impurity scattering is adopted with the Brooks-Herring model.
\revision{Injection boundary conditions \cite{Hong2009,Hong2011} are used for ohmic contacts.}
   Band gap narrowing and the Pauli principle are not considered.
   
\revision{In the spherical harmonics expansion,
the distribution function is expanded up to a predefined maximum order, $l_{max}$.
   When $l_{max}$ is too low, considerable errors can be obtained.
   For example, it has been demonstrated that the first-order expansion is not sufficient for small devices \cite{Rahmat1996,Jungemann2006}.
   Since an appropriate value of $l_{max}$ depends on the device structure and the bias point, it is important to check the convergence of the simulated quantities carefully \cite{Fischetti2016}. 
   In this work, the convergence of the simulated quantities (such as terminal currents and voltages) is verified by simulations with various $l_{max}$ values.}
      
   The strong electric field inside the device enforces introduction of a special stabilization scheme, such as the $H$-transformation \cite{Gnudi1993} or the maximum entropy dissipation scheme \cite{Ringhofer2002,Jungemann2006}.
   In particular, by virtue of its construction, the $H$-transformation requires interpolation of the previous solution variables under the present band profile, which inevitably results in interpolation errors.
   Therefore, the transient simulation capability that is compatible with the $H$-transformation is not currently available.  
   In this work, following the suggestion in \cite{Rupp2016}, the kinetic-energy-based scheme \cite{Jungemann2006} instead of the $H$-transformation has been adopted.
\revision{Developing a new stabilization scheme, which allows the transient simulation while taking advantages of the $H$-transformation, would be a formidable task.
   Progress on this issue will be reported elsewhere.}   
 
%   Throughout the following sections, the maximum order of the spherical %harmonics expansion, $l_{max}$, is 5, which gives quite accurate results %for the steady-state simulation results.

\subsection{Transient capability}

   With adoption of the kinetic-energy-based stabilization scheme, implementing the transient capability is straightforward. 

   When the entire Boltzmann equation is projected, the resultant Boltzmann equation reads \cite{Hong2009,Hong2011}
\begin{multline}
\label{eq_balance}
\sum_{l'}
\frac{\partial}{\partial t} \left[ Z_{l,l'} f_{l'} \right]
+ \frac{\partial}{\partial \varepsilon} \left[ \mathbf{F} \cdot \mathbf{A}_{l,l'} f_{l'} \right]
+ \nabla_{\mathbf{r}} \cdot \left[ \mathbf{A}_{l,l'} f_{l'} \right] \\
- B_{l,l'} f_{l'}
= \hat{S}_{l}^{in} - \hat{S}_{l}^{out},
\end{multline} 
where $l$ and $l'$ represent the index pairs for the spherical harmonics expansion, \revision{$t$ is the time,} $Z_{l,l'}$ is the density-of-states, $f_{l}$ is the projected distribution function, $\varepsilon$ is the kinetic energy, $\mathbf{F}$ is the force, $\mathbf{A}_{l,l'}$ and $B_{l,l'}$ are transport parameters \cite{Hong2011}, and $\hat{S}_{l}^{in}$ and $\hat{S}_{l}^{out}$ are scattering integrals.
   Since the discretization of all other terms except for the time derivation has been investigated extensively in previous works, the discretization of the time derivative term, $\frac{\partial}{\partial t} \left[ Z_{l,l'} f_{l'} \right]$, is the remaining task.

   An implicit time marching scheme is adopted to discretize the time derivative term.
   In an implicit time marching scheme, the time derivative of $Z_{l,l'} f_{l'}$ at a time instance $t_i$ can be written as 
\begin{equation}
\frac{\partial [ Z_{l,l'} f_{l'} ]}{\partial t} \Big|_{t=t_i} \approx Z_{l,l'}(\mathbf{r},\varepsilon) \sum_{j=0}^{N-1} a_{i,j} f_{l'}(\mathbf{r},\varepsilon,t_{i-j}),
\end{equation}
where $N$ is the order of the time-marching scheme and $a_{i,j}$ is a coefficient connecting $t_i$ and $t_{i-j}$. 
   Note that the density-of-states at given $(\mathbf{r},\varepsilon)$ is time-invariant in the kinetic-energy-based scheme. 

   It has been found that the first-order method (the backward Euler method) introduces excessively strong over-damping even with a fairly fine time step. 
   Therefore, two second-order methods are applied.
   When the periodic input signal is applied, 
the modified second-order method based on trigonometric functions \cite{Brachtendorf2013,Jungemann2017}, which is efficient for the periodic case, is used. 
   In this scheme, with a constant time step of $\Delta t$, the coefficients are given by
\begin{equation}
a_{i,0 } = \frac{1}{\Delta t} \frac{z\cos(2z) - z\cos(z)}{\sin(2z)-2\sin(z)},
\end{equation} 
\begin{equation}
a_{i,1} = \frac{1}{\Delta t} \frac{z \sin(z)}{\cos(z) - 1},
\end{equation}  
\begin{equation}
a_{i,2} = \frac{1}{\Delta t} \frac{z}{2\sin(z)},
\end{equation}    
where $z = \frac{2\pi}{n_T}$ and $n_T$ is the number of time steps per period.
   On the other hand, when the input signal is not periodic, 
following the second-order backward differentiation formula, 
the coefficients are given by
\begin{equation}
a_{i,0 } = \frac{1.5}{\Delta t},~~~ a_{i,1} = -\frac{2}{\Delta t},~~~ a_{i,2} = \frac{0.5}{\Delta t}. 
\end{equation} 
    
%/////////////////
%// Section III //
%/////////////////

\section{Numerical Results}

\revision{In this section, electrical properties of homogeneous samples and an N$^+$NN$^+$ structure are simulated by using the transient Boltzmann transport equation solver. 
   For some cases, the results of the drift-diffusion (DD) model are also shown for comparison.}
   The standard DD model neglects the time derivative of the current density, $\frac{\partial}{\partial t}\mathbf{J}_n$ \cite{Grasser2003}.
\revision{The current density is given by
\begin{equation}
\mathbf{J}_n = q \mu_n n \mathbf{E} + q D_n \nabla_{\mathbf{r}} n,
\end{equation}   
where $q$ is the absolute elementary charge, $\mu_n$ is the electron mobility, $n$ is the electron density, \revision{$\mathbf{E}$ is the electric field,} and $D_n$ is the diffusivity.
   In this work, the standard DD model is denoted as the ``QSJ'' (quasi-static current density) DD model.
   It is noted that the QSJ DD model still considers the time derivative of the electron density by solving the continuity equation.    
   On the other hand, the DD model which does not neglect $\frac{\partial}{\partial t}\mathbf{J}_n$ \cite{Jungemann2017} is called the ``full'' DD model.
   The equation for the current density reads
\begin{equation}
\mathbf{J}_n + \tau \frac{\partial}{\partial t}\mathbf{J}_n = q \mu_n n \mathbf{E} + q D_n \nabla_{\mathbf{r}} n,
\end{equation}  
where $\tau$ is the momentum relaxation time. 
}

\revision{Throughout this work, a uniform energy spacing of 10 meV is adopted.
   This value is found to be small enough from our numerical experience.}
   
\subsection{Homogeneous samples at equilibrium}

\revision{The donor density is 2$\times$10$^{17}$ cm$^{-3}$.
   In the DD model, the mobility ($\mu_n$) of 518 cm$^2$ V$^{-1}$ sec$^{-1}$ and the momentum relaxation time ($\tau$) of 86 fsec are used. 
   Since the DC bias voltage is zero, the DC electric field inside the sample vanishes.}
   A small excitation voltage is applied across the sample.
   After the initial transient effect is diminished, the current is recorded.
   By taking the Fourier coefficients of the current, the admittance is calculated.
   
\revision{In this example, an analytic solution for the admittance is readily available by solving the DD model. 
   In conjunction with the equation for the current density, the Poisson equation and the continuity equation should be solved.
   At equilibrium, the linearized equations at the angular frequency $\omega$ are given by
\begin{equation}
\label{eq_poisson}
-\epsilon \frac{\partial^2 }{\partial x^2} \delta \phi = -q \delta n,
\end{equation}   
\begin{equation}
j\omega q \delta n = \frac{\partial}{\partial x} \delta J_n,
\end{equation}
where $\epsilon$ is the permittivity, $x$ is the position, $\delta \phi(x)$ is the AC component of the electrostatic potential, $\delta n(x)$ is the AC component of the electron density, $j$ is the imaginary unit, $\delta J_n(x)$ is the AC component of the current density.
   In the case of the full DD model, all those equations are combined into
\begin{equation}
\label{eq_deltan}
\frac{\partial^2}{\partial x^2} \delta n = \left[ \frac{q n}{\epsilon V_T} + \frac{j\omega (1+j\omega\tau)}{D_n}  \right] \delta n,
\end{equation}   
where $V_T$ is the thermal voltage and the Einstein relation is employed.
   It is noted that the term $\omega \tau$ should be neglected in the QSJ DD model.
   By integrating the total current density (which is a sum of $\delta J_n$ and the displacement current density) over the sample, the AC current, $\delta I$, is readily obtained as
\begin{equation}
\frac{\delta I}{A} = \frac{1}{1+j \omega \tau} q \mu_n \left[ n \frac{\delta V}{L} - V_T \frac{\delta n(L)-\delta n(0)}{L} \right] + j \omega \epsilon \frac{\delta V}{L},
\end{equation}
where $A$ is the device area (1.0 $\mu$m$^2$ in this example) and $\delta V$ is the AC voltage.
}

\revision{When the ideal ohmic boundary condition is considered,
\begin{equation}
\label{eq_idealbc}
\delta n(0) = 0,~~~ \delta n(L) = 0,
\end{equation}
only the trivial solution of (\ref{eq_deltan}) exists.   
   The admittance is then analytically given as 
\begin{equation}
\label{eq_analytic}
Y(\omega) = \lim_{\delta V \to 0} \frac{\delta I}{\delta V} = \frac{1}{1+j \omega \tau} q \mu_n n \frac{A}{L} + j \omega \epsilon \frac{A}{L}.
\end{equation}
   At high frequencies, the capacitive contribution from the second term dominates. 
   At low frequencies, in the case of the full DD model, the susceptance (the imaginary part of the admittance) may or may not become negative, depending on the sign of $\tau q \mu_n n - \epsilon$.
   The negative susceptance is originated from the time derivative of the current density.
   For a relatively long sample, a reasonable agreement between the results from the Boltzmann transport equation and (\ref{eq_analytic}) is achieved as shown in Fig. \ref{fig_homo_y_long}.
   The sample length is 1200 nm.         
}

\begin{figure}[!t]
\centering
\includegraphics[width=3in]{homo_y_long.jpg}
\caption{\revision{Admittance of a homogeneous sample at equilibrium.
The sample length is 1200 nm.
It is obtained by solving the Boltzmann transport equation (the full BTE model).
The results of an analytic equation, (\ref{eq_analytic}), are shown for comparison.}}
\label{fig_homo_y_long}
\end{figure} 
  
\revision{However, for a relatively short sample, the boundary condition has more significant impact on the simulation results. 
   Fig. \ref{fig_homo_y} shows the admittance of a 120-nm-long sample at equilibrium.
   The result from the Boltzmann transport equation shows that the simple analytic expression with the ideal boundary condition, (\ref{eq_analytic}), fails to predict the admittance even at low frequencies.
   In order to see the effect of the boundary condition, the admittance is also calculated with the equilibrium distribution applied to ohmic contacts (empty symbols in the figure).
   It corresponds to the ideal ohmic boundary condition, (\ref{eq_idealbc}).
   As shown in Fig. \ref{fig_homo_y}, the admittance with the ideal ohmic boundary condition agrees well with (\ref{eq_analytic}).}

\revision{Using the DD model, the reason of discrepancy can be explained as follows.
   With the recombination velocity of contacts, $v_{rec}$, more realistic boundary conditions for two contacts are written as 
\begin{equation}
\delta J_n(0)= q \delta n(0) v_{rec},
\end{equation}
\begin{equation}
\delta J_n(L)= -q \delta n(L) v_{rec}.
\end{equation}
   Then, a non-trivial solution, $\delta n$, of (\ref{eq_deltan}) can be found.
   Therefore, because of non-vanishing $\delta n(L) - \delta n(0)$, the admittance can be significantly deviated from (\ref{eq_analytic}).
   In Fig. \ref{fig_homo_y}, the full DD results with two different values of $v_{rec}$ are also shown.
   With a certain value of the recombination velocity ($9\times10^{6}$ cm sec$^{-1}$ in this example), excellent agreement with the full BTE model is obtained.
   When the recombination velocity is much lower than the optimal value (for example, $4\times10^{6}$ cm sec$^{-1}$ in the figure), the low-frequency conductance is reduced and the low-frequency susceptance is overestimated.}
   This reveals that not only the correct transport equation but also the correct boundary condition are critically important to gain the correct admittance.
   
\begin{figure}[!t]
\centering
\includegraphics[width=3in]{Fig2_Revision.jpg}
\caption{Admittance of a homogeneous sample at equilibrium.
\revision{The sample length is 120 nm.}
It is obtained by solving the Boltzmann transport equation (the full BTE model).
\revision{Impact of the ideal ohmic boundary condition on the admittance is clearly shown.
The full drift-diffusion (DD) model results with two different recombination velocities are compared.}}
\label{fig_homo_y}
\end{figure}   
      
   The time derivative of the current density is usually neglected in the conventional device simulation.
   In order to examine the impact of the time derivative term, two different simulation sets are generated.
   In addition to the simulation with correct implementation of the time derivative term (the ``full'' BTE model), an approximated implementation (the ``QSJ'' BTE model) is tried.
   In the QSJ BTE model, the time derivative term is implemented only for the zeroth-order distribution function, while it is completely ignored for the higher-order distribution functions.
   The results shown in Fig. \ref{fig_homo_y_qsj} show that the approximation yields significant errors in the admittance.
   Even in terms of the overall magnitude, a significant error is observed at frequencies of around two or three THz.
%\revision{Although the susceptance is reduced as the frequency is decreased, %the results from two BTE models do not converge.}
         
\begin{figure}[!t]
\centering
\includegraphics[width=3in]{homo_y_qsj.jpg}
\caption{Admittance of a homogeneous sample at equilibrium.
In the QSJ BTE model, the time derivative of the distribution function is considered only at the zeroth-order.}
\label{fig_homo_y_qsj}
\end{figure}    

\subsection{N$^+$NN$^+$ structure}

   An N$^+$NN$^+$ structure whose length is 120 nm is simulated.     
   The device area is 1.0 $\mu$m$^2$.  
   The 40-nm-long lowly doped region is surrounded by two highly doped regions.
\revision{The DC IV curve is shown in Fig. \ref{fig_nnn_dc}(a).}
   As the maximum order of the spherical harmonics expansion, $l_{max}$, increases, the anode current is converged.
\revision{In Fig. \ref{fig_nnn_dc}(b), the doping profile and the electron velocity at the anode voltage of 0.5 V are shown as functions of the position.
   The maximum velocity is much larger than the saturation velocity, due to the strong quasi-ballistic transport.} 
   Since the anode current is already converged with $l_{max}$ = 5, 
the transient simulation also has been performed with $l_{max}$ = 5.    
   
\begin{figure}[!t]
\centering
\includegraphics[width=3.5in]{Fig4_Revision.jpg}
\caption{\revision{(a) DC IV curve of a 120-nm-long structure.
$l_{max} =$ 1, 3, and 5. 
(b) Doping profile and electron velocity as functions of the position. The anode voltage is 0.5 V.}}
\label{fig_nnn_dc}
\end{figure}     
   
   Small-signal anode currents of the 120-nm-long structure for the first five periods are shown in Fig. \ref{fig_nnn_tr}. 
   The DC voltage is 0.5 V and various frequencies are tested. 
\revision{At low frequencies, the conductive components dominate and the phase of the current is very similar to that of the applied voltage.}
   As the frequency increases (for example at 10 THz), the capacitive contribution becomes significant.
\revision{Such a capacitive contribution exhibits the phase difference of 90 degrees.}
      
\begin{figure}[!t]
\centering
\includegraphics[width=3in]{nnn_tr.jpg}
\caption{Transient simulation results for the anode current. 
The small-signal response without the DC current is drawn. 
The amplitude of the input signal is 1 mV.
The 120-nm-long structure is considered. 
The DC voltage is 0.5 V and various frequencies are tested. 
Time is normalized to the period.}
\label{fig_nnn_tr}
\end{figure}   

   The small-signal admittance 
is shown for various excitation frequencies in Fig. \ref{fig_nnn_y}.
   Below a few hundred GHz, the conductance (the real part of the admittance) does not change significantly. 
   Reduction of the conductance at 0.5 V originates from the DC IV characteristics shown in Fig. \ref{fig_nnn_dc}.  
   Beyond 5 THz, the admittance does not depend on the bias voltage.
%   At high frequencies, inclusion of the time derivative in the odd equations %introduces much difference in the small-signal admittance. 
%   Moreover, the drift-diffusion simulation gives a result which is very %similar to that without the time derivative in the odd equations. 
%   Therefore, it is confirmed that the missing time derivative in the current %density equation is important at high frequencies.  

\begin{figure}[!t]
\centering
\includegraphics[width=3in]{nnn_y.jpg}
\caption{Frequency dependence of the small-signal admittance calculated with the full BTE model.  
%Calculation results with different approximations for the time derivative %term are compared. 
Two DC bias voltages of 0.0 V and 0.5 V are considered.}
\label{fig_nnn_y}
\end{figure}  
    
   Thus far, the small-signal response has been calculated by the transient simulation.
   However, the transient Boltzmann transport equation solver can be applied to general large-signal cases.
   The large-signal response is shown in Fig. \ref{fig_nnn_ls}.    
   The amplitude of the sinusoidal excitation is as large as 0.5 V.
   The frequency is 3 THz.
   As shown in the figure, the large-signal excitation with the peak-to-peak amplitude of 1 V can be simulated without numerical problems.      
   This demonstrates the robustness of the deterministic Boltzmann equation solver.  
   The third order harmonic component (9 THz) of the anode current is about 0.8 $\%$ of the fundamental one (3 THz). 
   Higher order harmonics are generated by the nonlinearity of the device, which cannot be captured by the small-signal analysis.
   Also, comparison with the QSJ simulation results reveals the importance of the time derivative term.
%   For a nonlinear system at a high operation frequency, the deterministic %Boltzmann equation solver developed in this work can be used as a valuable %tool.   
%   High frequency components in the anode current can be clearly observed.\
%   It is due to the nonlinear current-voltage characteristics of the N$^+$NN$^+$ %structure. 
   
\begin{figure}[!t]
\centering
\includegraphics[width=3in]{nnn_ls.jpg}
\caption{\revision{Large-signal simulation results of the anode current. The frequency of the voltage excitation is 3 THz.}
The peak-to-peak amplitude of the input signal is 1 V.}
\label{fig_nnn_ls}
\end{figure}  

   Next, the step response of the  N$^+$NN$^+$ structure is simulated. 
\revision{The anode bias voltage is increased from 0 V to 0.5 V with a smooth waveform of $0.25 - 0.25 \cos(\frac{\pi t}{t_{ramp}})$ V, where $t_{ramp}$ is the ramping time.
   Once after the voltage reaches 0.5 V, it does not change.}
   Various ramping speeds have been tested.
   The duration of the ramping phase is used to normalize the time variable.
   For slow ramping rates, the results are similar with the quasi-static result as shown in Fig. \ref{fig_nnn_step}.
   However, for faster ramping rates, the anode current is heavily deviated from the quasi-static result. 
   For cases with large average ramping rates such as 5 V psec$^{-1}$, strong overshoot of the anode current is observed.   
\revision{In those cases, the displacement current contributes significantly to the total current as shown in Fig. \ref{fig_nnn_step}.}
   Even after the ramping period is finished, the anode current takes a considerable amount of time to be relaxed. 

\begin{figure}[!t]
\centering
\includegraphics[width=3in]{nnn_step.jpg}
\caption{Step-response of the anode current. 
\revision{In addition to the total currents (solid lines), the displacement current with the ramping time oaf 0.1 psec (a dashed line) is also shown.}
The time variable is normalized with the ramping time.
Various ramping rates are tested.}
\label{fig_nnn_step}
\end{figure}  

   The time evolution of the electron distribution function is plotted in Fig. \ref{fig_nnn_step_distribution}.    
   The ramping rate is 5 V psec$^{-1}$ and the elapsed times are $t$ = 0.1 psec (just after the ramping period) and 0.2 psec (0.1 psec elapsed after the ramping period).
   At $t$ = 0.1 psec, the electron distribution function (its zeroth-order expansion coefficient) is close to that in the equilibrium, because it takes a finite time to heat the electron gas.
   On the other hand, at $t$ = 0.2 psec, it is clearly shown that high-energy electrons are generated near the anode terminal. 

\begin{figure}[!t]
\centering
\includegraphics[width=3.5in]{nnn_step_distribution.jpg}
\caption{Electron distribution function for time points, (a) $t$ = 0.1 psec and (b) $t$ = 0.2 psec.
The ramping rate is 5 V psec$^{-1}$.
}
\label{fig_nnn_step_distribution}
\end{figure} 
   
%////////////////
%// Section IV //
%////////////////

\section{Application to Plasma Effects in a MOSFET}

\revision{Using the transient simulation capability,
the detector operation of a scaled MOSFET is simulated.}
   It is an ideal tool to estimate the plasma modes in the two-dimensional electron gas.
\revision{Previous studies on plasma effects in the scaled devices via full-band Monte Carlo simulations can be found in \cite{Fischetti2001,Sano2011}.
}

   A double-gate MOSFET, whose oxide thickness is 1 nm, is considered. 
   The silicon substrate is 10 nm.
   Only the upper half of the entire structure has been simulated.
   The doping profile along the channel direction is identical to that of the N$^+$NN$^+$ structure shown in the previous section.
   The gate length is 60 nm and the gate workfunction is 4.5 eV.
    
   The one-dimensional electron transport is considered, while the two-dimensional Poisson equation is solved together \cite{Hong2015}. 
\revision{No quantum confinement effect is included in this simulation.
   Therefore, the finite thickness of the electron gas \cite{Dahl1977,Fischetti2001a} is not considered.
   In order to evaluate the impact of this effect, the multisubband approach \cite{Jin 2008,Ruic2015} would be required, which is an interesting future research topic.}
%Although no quantum confinement effect is included in this simulation, it %is expected that the general trend of the results is valid.
   The surface-roughness scattering mechanism is neglected.
   Only the bulk phonon and impurity scattering mechanisms are considered.
   The input characteristics at the steady-state condition are shown in Fig. \ref{fig_mos_dc}. 
   The threshold voltage is about 0.3 V.
   It is also confirmed that the simulation with $l_{max}$ = 5 gives reasonably converged values. 

\begin{figure}[!t]
\centering
\includegraphics[width=3in]{mos_dc.jpg}
\caption{Input characteristics of the 60-nm MOSFET at the steady-state condition. The drain voltage is fixed to 0.01 V.
The surface-roughness scattering is neglected.
}
\label{fig_mos_dc}
\end{figure}  

   The detector operation \cite{Dyakonov1996} is simulated.
   The linear operation mode (so called ``cold FET'') has been exclusively considered in the simulation.
\revision{The source terminal is directly connected to the ground and the drain terminal is electrically open.
   Therefore, no current is allowed at the drain terminal.
   However, it does not necessarily mean that the drain voltage also vanishes.
   Actually, when an input signal is applied to the gate terminal, a non-vanishing drain voltage is induced so that the drain current vanishes.
   The DC component of the induced drain voltage is often used to quantify the detector response \cite{Dyakonov1996}.}

   Once after the steady-state solution is obtained, the input signal, whose voltage amplitude is 1 mV, is applied to the gate terminal.  
   A figure-of-merit, the voltage responsivity, $R_V$, is defined as
\begin{equation}
R_V =  \frac{\left| V_D \right|}{P_{AC}},
\end{equation}   
where $V_D$ is the induced DC drain voltage and $P_{AC}$ is the dissipated AC power of the MOSFET.  
\revision{The amplitude of the AC gate voltage does not change the responsivity because the responsivity is normalized with $P_{AC}$.
   It has been also numerically verified by the simulation results with different amplitudes.}
   Since the two-dimensional cross section is simulated, $R_V$ has to be divided by the width of the device in the third dimension \cite{Jungemann2017}.
  
   As in the previous section, the transient simulation has been performed.
   After the initial transient effect is diminished, the DC drain voltage is extracted.
   Two simulation sets (the full BTE and QSJ BTE models) are tested.
\revision{Only when the time derivative of the distribution function is considered (the full BTE model),
the plasma mode is obtained by the simulation.
   In \cite{Jungemann2017}, it has been shown that even the small-signal admittance obtained by the full model can significantly differ from the one obtained by the QSJ model.
   The difference of the small-signal response also affects the induced drain voltage.
   Therefore, the values of $R_V$ obtained from the full and QSJ models may differ. 
   The difference between two models is interpreted as an indicator for the impact of the plasma mode.}

\revision{The voltage responsivity is shown as a function of the frequency in Fig. \ref{fig_mos_responsivity_vg}.
   Two gate bias voltages are tested.}
   The responsivity shows a maximum value near the threshold voltage \revision{(0.3 V)}.
\revision{At this bias condition, the two models give almost identical results.
   Therefore, significant impact of the plasma effect cannot be observed.}
   However, above the threshold voltage \revision{(1.0 V)}, the full BTE results slightly deviate from the QSJ ones.
\revision{Local peaks are observed at certain frequencies only in the full BTE result.}
   %The sign of $V_D$ changes from negative to positive at about 1.9 V.
   Such differences are attributed to the plasma modes in the channel.  
\revision{It is noted that, depending on the frequency, the full BTE results can be larger or smaller than the QSJ results \cite{Jungemann2017}.}   
        
\begin{figure}[!t]
\centering
\includegraphics[width=3in]{mos_responsivity_freq.jpg}
\caption{\revision{Absolute value of the responsivity, $R_V$, as a function of the frequency. Two gate bias voltages (0.3 V and 1.0 V) are tested. 
The responsivities at 1.0 V are scaled for clarity.
Local peaks are pointed out by arrows.
The induced DC drain voltage, $|V_D|$, is also shown in the right $y$-axis.
The amplitude of the AC gate voltage is 1 mV.}
}
\label{fig_mos_responsivity_vg}
\end{figure} 

%   Frequency dependence of $R_V$ up to 20 THz is shown in Fig. \ref{fig_mos_responsivity_freq}.
%   Two gate bias voltages of 0.25 V (in the subthreshold region) and 1.0 %V (above the threshold voltage) are selected.
%   The QSJ BTE results are also shown.
%   The difference between two models is interpreted as a result of the plasma %effect.  
   
%\begin{figure}[!t]
%\centering
%\includegraphics[width=3in]{mos_responsivity_freq.jpg}
%\caption{Absolute value of the responsivity, $R_V$, as a function of the %frequency. 
%The gate voltages are 0.25 V and 1.0 V.}
%\label{fig_mos_responsivity_freq}
%\end{figure}    

   It has been found that the voltage responsivity of the 60-nm MOSFET is only slightly affected by the plasma effect.
   It is expected that the plasma modes in the electron gas will have a larger impact in aggressively scaled devices.
   Systematic investigation of the detector performance is of great research interest, and will be reported elsewhere.  

%   One question of interest is whether the drift-diffusion model can predict %this behavior correctly. 
%   Although the drift-diffusion model cannot be generally valid, at bias %conditions close to the zero drain voltage, a certain agreement is expected.
%   In this work, the mobility and the momentum relaxation time are taken %from the DC BTE result.
%   Using the admittance at a low frequency, the surface recombination velocity %is fitted.
%   It has been found that the surface recombination velocity is more and %less constant (about 6$\times$10$^{6}$ cm sec$^{-1}$) over a wide range %of the gate voltage. 
%   From Fig. \ref{fig_mos_responsivity_dd}, it is concluded that the drift-diffusion %model (with $\frac{\partial}{\partial t}J_n$) can generate a qualitatively %sound result as much as the mobility is calibrated against the BTE result.
%   Of course, at nonzero DC bias current, the validity of the drift-diffusion %model is questionable.  
  
%   In this work, the linear operation mode (so called ``cold FET'') has %been exclusively considered in the simulation.
%   Since it is implemented in a general-purpose device simulator, application %to the saturation mode is also possible.
%   Investigation on the device parameters for the stable plasma oscillation %will be reported elsewhere.

%///////////////
%// Section V //
%///////////////

\section{Conclusions}

   In this work, the transient simulation has been performed using a deterministic Boltzmann equation solver. 
   A kinetic-energy-based scheme is adopted with sufficiently dense grids.
   Implicit time marching methods are used.
   
   It has been demonstrated that the transient simulation using a deterministic Boltzmann equation solver is possible.
\revision{Examples for homogeneous samples, an N$^+$NN$^+$ structure, and a MOSFET have been shown.}
   The importance of the boundary condition is also shown.
   Furthermore, considering the time derivative of the distribution completely (the full BTE model in this work) is mandatory for accurate calculation in high frequencies. 
   
% use section* for acknowledgment
\section*{Acknowledgment}

   The authors would like to thank Prof. Christoph Jungemann (RWTH Aachen, Germany) and Prof. Akira Satou (Tohoku University, Japan) for helpful discussions.
%\revision{Valuable comments from the reviewers are gratefully appreciated.}

% Can use something like this to put references on a page
% by themselves when using endfloat and the captionsoff option.
\ifCLASSOPTIONcaptionsoff
  \newpage
\fi



% trigger a \newpage just before the given reference
% number - used to balance the columns on the last page
% adjust value as needed - may need to be readjusted if
% the document is modified later
%\IEEEtriggeratref{8}
% The "triggered" command can be changed if desired:
%\IEEEtriggercmd{\enlargethispage{-5in}}

% references section

% can use a bibliography generated by BibTeX as a .bbl file
% BibTeX documentation can be easily obtained at:
% http://mirror.ctan.org/biblio/bibtex/contrib/doc/
% The IEEEtran BibTeX style support page is at:
% http://www.michaelshell.org/tex/ieeetran/bibtex/
%\bibliographystyle{IEEEtran}
% argument is your BibTeX string definitions and bibliography database(s)
%\bibliography{IEEEabrv,../bib/paper}
%
% <OR> manually copy in the resultant .bbl file
% set second argument of \begin to the number of references
% (used to reserve space for the reference number labels box)
% 
\bibliographystyle{IEEEtran}
\bibliography{SDSL_bib}

% biography section
% 
% If you have an EPS/PDF photo (graphicx package needed) extra braces are
% needed around the contents of the optional argument to biography to prevent
% the LaTeX parser from getting confused when it sees the complicated
% \includegraphics command within an optional argument. (You could create
% your own custom macro containing the \includegraphics command to make things
% simpler here.)
%\begin{IEEEbiography}[{\includegraphics[width=1in,height=1.25in,clip,keepaspectratio]{mshell}}]{Michael Shell}
% or if you just want to reserve a space for a photo:

%\begin{IEEEbiography}{Michael Shell}
%Biography text here.
%\end{IEEEbiography}

% if you will not have a photo at all:
%\begin{IEEEbiographynophoto}{John Doe}
%Biography text here.
%\end{IEEEbiographynophoto}

% insert where needed to balance the two columns on the last page with
% biographies
%\newpage

%\begin{IEEEbiographynophoto}{Jane Doe}
%Biography text here.
%\end{IEEEbiographynophoto}

% You can push biographies down or up by placing
% a \vfill before or after them. The appropriate
% use of \vfill depends on what kind of text is
% on the last page and whether or not the columns
% are being equalized.

%\vfill

% Can be used to pull up biographies so that the bottom of the last one
% is flush with the other column.
%\enlargethispage{-5in}



% that's all folks
\end{document}


